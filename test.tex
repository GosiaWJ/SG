\documentclass{article}
\usepackage[utf8]{inputenc}
\usepackage[polish]{babel}
\usepackage{polski}
\usepackage{enumerate}
\usepackage{natbib}
\usepackage{graphicx}
\usepackage{geometry}
\newgeometry{tmargin=2.5cm, bmargin=2.5cm, lmargin=2.5cm, rmargin=2.5cm}
%wersja wyslana na eportal, mmm
\makeatletter
\newcommand{\linia}{\rule{\linewidth}{0.4mm}}
\renewcommand{\maketitle}{\begin{titlepage}
    \vspace*{1cm}
    \begin{center}
    Politechnika Wrocławska\\
    AiR ARR\\
 Projekt zespołowy
    \end{center}
      \vspace{3cm}
    \begin{center}

     \LARGE \textsc {\@title}
         \end{center}
     \vspace{1cm}

    \begin{center}
    \textit{ Autorzy:}\\
   \textit{\@author}
     \end{center}
      \vspace{1cm}

     \begin{center}

    Prowadzący:
  dr inż. Krzysztof Arent %dorobić inż mgr itd
    \end{center}

    \vspace*{\stretch{6}}
    \begin{center}
    \@date
    \end{center}
  \end{titlepage}
}
\makeatother
\author{Beata Berajter\\
Dawid Brząkała\\
Dorota Gidel\\
Katarzyna Wądrzyk\\
Ada Weiss\\
Małgorzata Witka-Jeżewska\\
 }%wpisać indeks
\title{SensGlove}

\begin{document}

\maketitle
\newpage
\tableofcontents
\newpage
\section{Opis projektu}
\subsection{Wstęp}
Celem projektu jest zbudowanie stanowiska do zbierania Bazy Danych biosygnałów oraz sygnałów z rękawiczki sensorycznej wchodzącej w interakcję z przedmiotami. Podjęcie tej tematyki umożliwi dalsze prace nad protezami kończyn górnych, w szczególności dłoni. Wyniki projektu wspomogą prace prowadzone nad protezami rąk, które ułatwiają wykonywanie codziennych czynności osobom niepełnosprawnym. Ważnym jest, aby proteza przy poruszaniu się przypominała prawdziwą kończynę w jak największym stopniu. Osiągnąć to można poprzez tworzenie bazy danych gdzie umieszczane będą interakcje palców ręki z różnymi przedmiotami codziennego użytku.
Badania te mogą zostać użyte nie tylko przy nowoczesnych protezach, lecz również przy budowie nowych, sprawniejszych robotów humanoidalnych.\\
Pierwszym krokiem przy realizacji projektu jest zapoznanie się z istniejącym już stanowiskiem do pomiarów, które umiejscowione jest na Politechnice Wrocławskiej, budynek C-3, sala 06. Po dogłębnym zaznajomieniu się z istniejącym już oprogramowaniem wykonamy nasze własne stanowisko badawcze, które składać się będzie z rękawiczki sensorycznej podłączonej poprzez mikrokontroler do karty, do której trafiają równocześnie pobierane biosygnały.\\
Efektem końcowym będzie stanowisko do poszerzania bazy danych zawierającej biosygnały oraz sygnały charakteryzujące interakcje palców protezy z przedmiotem.\\
Wyniki projektu będą upowszechniane przy pomocy strony internetowej (http://sensglove.happyrobotics.com/).\\



\end{document}
